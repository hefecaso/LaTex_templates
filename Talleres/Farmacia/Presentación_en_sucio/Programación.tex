\documentclass[12pt,letterpaper]{article}
\usepackage[utf8]{inputenc}
\usepackage[spanish]{babel}
\usepackage{amsmath}
\usepackage{amsfonts}
\usepackage{amssymb}
\usepackage{makeidx}
\usepackage{graphicx}
\usepackage[left=3cm,right=1.5cm,top=1.5cm,bottom=1.5cm]{geometry}
\usepackage{enumerate}

\title{\LaTeX \hspace{0.1cm} en farmacia}
\author{Departamento de química orgánica}
\date{\today}

\begin{document}
\maketitle

\section{Lista de temas:}

\subsection{Un poco de teoría}

\begin{enumerate}
    \item ¿Qué es Latex?
    \item ¿Por qué usar Latex?
    \item Ventajas de Latex sobre Word
\end{enumerate}

\section{Parte práctica}

\begin{enumerate}
    \item Creando un nuevo documento.
    \item Uso de los paquetes.
    \item Añadiendo a varios autores.
    %--------------------------------------------------
    \item Colocando tablas
        \begin{enumerate}
            \item Colocando una tabla normal con tamaños de celdas definidas.
            
            \item Definiendo diferentes parámetros de alineaciones al colocar texto en las tablas.
            
            \item Parámetros para fijar una tabla.
            
            \item Colocar una figura dentro de una tabla.
        \end{enumerate}
    %--------------------------------------------------
    \item Ecuaciones
        \begin{enumerate}
            \item Como insertar ecuaciones dentro de un texto y fuera del mismo.
            \item Alineación y Enumeración de ecuaciones.
            \item Fracciones.
            \item Raíces y Exponentes
            \item Signos de Agrupación
        \end{enumerate}
    %--------------------------------------------------
    \item Iniciando con la recopilación de un reporte, proporcionado por la escuela de farmacia. Autores originales: (Tzoc et al., 2020).
    
    \item Insertando secciones a dos columnas en un documento de una columna.
    
    \item Explicando las subsection, subsubsection y subsubsubsection.
    
    \item Incertando items.
    %--------------------------------------------------
    \item Regresando a una columna el documento.
    
    \item Explicando como insertar figuras y sus configuraciones, importancia de los label y los caption.
    
    \item Explicando como realizar anotaciónes, pié de página, referencia a figuras y tablas, citar, url y href.
    
    \item Explicando enumerate y configuraciónes.
    
    \item Insertar bibliografía
    
    \item Como citar una bibliografía.
\end{enumerate}


\end{document}
