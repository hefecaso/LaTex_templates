%[Tamaño de letra, tipo de hoja, número de columnas]{tipo de documento}
\documentclass[12pt,letterpaper, onecolumn]{article}



%Decodificación para el lenguaje español y caracteres especiales.
\usepackage[utf8]{inputenc} 

%Definimos que trabajaremos en idioma español y el tipo de letra.
\usepackage[spanish]{babel}

%mejoras para mejorar la estructura de la información y la salida impresa de documentos que contienen fórmulas matemáticas.
\usepackage{amsmath}

%El paquete amsfonts brinda una colección de fuentes TeX adicionales diseñadas para su uso en material matemático con AmS-TeX.
\usepackage{amsfonts}

%El paquete amssymb permite usar símbolos especiales.
\usepackage{amssymb}

%Proporciona un índice ordenado a partir de datos sin procesar sin clasificar.
\usepackage{makeidx}

%El paquete se basa en el paquete de gráficos , proporcionando una interfaz clave-valor para argumentos opcionales para el comando \ includegraphics .
\usepackage{graphicx}

%El paquete float es importante para las imágenes con la opción [H] para que las imágenes se coloquen en donde lo deseamos
\usepackage{float}

%A continuación vamos a definir los márgenes de nuestro documento con geometry
\usepackage[left=3cm,right=1.5cm,top=1.5cm,bottom=1.5cm]{geometry}

%Este paquete permite cambiar las formas de enumerar items.
\usepackage{enumerate}

%<<<<<<<<<<<<<<<<<<<<  >>>>>>>>>>>>>>>>>>>
%Agregaremos el siguiente paquete para cambiar el número de columnas (El menor número será el que definamos al inicio del documento)
\usepackage{multicol}

%\begin{multicols}{2}

%\end{multicols}
%<<<<<<<<<<<<<<<<<<<<  >>>>>>>>>>>>>>>>>>>

%A continuación veremos como podemos agregar un encabezado en nuestro documento:
%>>>>>>>>>>Forma para un solo autor
%\title{\LaTeX \hspace{0.1cm} en farmacia}
%\author{Departamento de química orgánica}
%\date{\today}

%>>>>>>>>>>Forma para varios autores

%Primero incluimos el paquete authblk: El paquete redefine el comando \ author para que funcione normalmente o para permitir un estilo de nota al pie de entrada de autor / afiliación.
\usepackage{authblk}


% El comando \ balance se puede utilizar para equilibrar las columnas en la página final si se desea. Debe colocarse en cualquier lugar dentro de la primera columna de la última página.
\usepackage{balance}

%\hspace es para espacio horizontales
\title{Presentaciónes}
\author{----}
\date{\today}



%Con esto agregamos la fecha.
\date{\today}

%Paquete para configurar medidas de las tablas
\usepackage{tabularx}

%Para colocar hiperlinks \href{url}{text}
\usepackage[pdftex]{hyperref}

%<<<<<<<<< para saltos de página usar  \clearpage >>>>>
%<<<<<<<<< para saltos entre líneas usar \vspace{2cm}>>>>>

%<<<<<<<<< para espaciado horizontal \hspace{1cm}>>>>>

%<<<<<<<<< para colocar url o referencias a url usar \url{http://www.latex-project.org/} o  \href{http://www.latex-project.org/}{latex project}>>>>>>>


\begin{document}
\maketitle

\section{Werner Omar Chanta Bautista}

Licenciado en Física Aplicada por la Universidad dr San Carlos, Maestría en Física Médica por la Universidad Nacional de Costa Rica



\section{Carmen Lucía Vásquez Maldonado}

Estudiante de quinto año de la carrera de química, auxiliar del Departamento de Química Orgánica

\section{Federico Tzunux Tzoc}

Estudiante de quinto año de la carrera de química, auxiliar del área Fisicomatemática.

\section{Luis Armando Colorado Sequén}
Estudiante de décimo semestre de la carrera de Ingeniería Mecánica Eléctrica, miembro de la rama estudiantil de ingenieros eléctricos y electrónicos - USAC y actual vicepresidente del capitulo estudiantil de Electron Devices Society.



\section{Gerson Alexander Cux García}

Estudiante del noveno semestre de ingeniería electrónica, miembro de la rama estudiantil de ingenieros eléctricos y electrónicos - USAC y actual presidente del capitulo estudiantil de Electron Devices Society.


\section{Héctor Fernando Carrera Soto}


Estudiante del octavo semestre de ingeniería electrónica, miembro de la rama estudiantil de ingenieros eléctricos y electrónicos - USAC y actual presidente del capitulo estudiantil de Robotics and Automation Society.






\end{document}