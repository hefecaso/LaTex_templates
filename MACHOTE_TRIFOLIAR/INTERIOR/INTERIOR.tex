\documentclass[landscape]{article}
\usepackage {lscape} 
\usepackage[letterpaper,margin=1.5cm]{geometry}
\usepackage{lipsum,multicol}
\begin{document}

\begin{multicols}{3}

\subsection*{¿Qu\'e es el mantenimiento para PC?}

Es el cuidado que se le da a la computadora para prevenir posibles fallas, se debe tener en cuenta la ubicaci\'on f\'isica del equipo ya sea en la oficina o en el hogar, as\'i como los cuidados especiales cuando no se est\'a usando el equipo.

\subsection*{¿Por qu\'e es necesario limpiar una computadora?}

La mezcla de polvo con el ambiente h\'umedo en casos extremos ocasiona que este pueda ser un magnifico conductor electr\'onico provocando pequeñas fallas en los componentes electr\'onicos de una computadora; adem\'as de que la acumulaci\'on del mismo reduce la eficiencia de los ventiladores de enfriamiento, por otra parte, el polvo cuando se acumula de forma uniforme sobre los circuitos integrados forma un manto aislante el cual retien el calor provocando que los circuitos disminuyan su rendimiento.

 \subsection*{Mantenimiento preventivo para PC}

El mantenimiento preventivo consiste en crear un ambiente favorable para el sistema y conservar limpias todas las partes que componen una computadora. El mayor n\'umero de fallas que presentan los equipos es por la acumulaci\'on de polvo en los componentes internos, ya que \'este act\'ua como aislante t\'ermico.

Si se quiere prolongar la vida \'util del equipo y hacer que permanezca libre de reparaciones por muchos a\~nos se debe realizar la limpieza con frecuencia.

\subsection*{Mantenimiento correctivo para PC}

Consiste en la reparaci\'on de alguno de los componentes de la computadora, puede ser una soldadura peque\~na, el cambio total de una tarjeta (sonido, video, entre otras), o el cambio total de alg\'un dispositivo perif\'erico como el rat\'on, teclado, monitor, etc.

\columnbreak


\subsection*{Criterios que se deben considerar para el mantenimiento a la PC}

La periodicidad que se recomienda para dalre mantenimiento a la PC es de una vez por semestre, esto quiere decir que como m\'inimo debe d\'arsele dos veces al a\~no. Pero eso depender\'a de cada usuario de la ubicaci\'on y uso de la computadora, as\'i como de los cuidados adicionales que se le dan a la PC.

Por su parte, la ubicaci\'on f\'isica de la computadora en el hogar u oficina afectar\'a o beneficiara a la PC, por lo que deben tenerse en cuenta varios factores:

* No exponer a la PC a los rayos del Sol.

* No colocar a la PC en lugares h\'umedos.

* Mantener a la PC alejada de equipos electr\'onicos o bocinas que produzcan campos magn\'eticos ya que pueden da\~nar la informaci\'on.

* Limpiar con frecuencia el mueble donde se encuentra la PC as\'i como aspirar con frecuencia el \'area si es que hay alfombras.

* No fumar cerca de la PC.

* Evitar comer cuando se est\'e usando la PC.

*Usar "UPS" para regular la energ\'ia el\'ectrica y por si la energ\'ia se corta que haya tiempo de guardar la informaci\'on.

* Cuando se deje de usar la PC, esperar a que se enfr\'ie el monitor y ponerle una funda protectora, as\'i como al teclado y al chasis del CPU.

* Revisi\'on de la instalaci\'on el\'ectrica de la casa u oficina, pero esto lo debe de hacer un especialista.


\subsection*{Mantenimiento preventivo a dispositivos}

\subsubsection*{Monitor}

En ning\'un momento cuando se habla de mantenimiento preventivo, se debe de pensar en que se va a abrir el monitor para limpiarlo. El monitor contiene condensadores de alta capacidad el\'ectrica que pueden producir un peligroso y hasta mortal choque el\'ectrico incluso despu\'es de haberlo apagado y desconectado.

En vez de ello, hay que concentrarse en limpiar el exterior del monitor y la pantalla.
\columnbreak

Generalmente se ocupa una buena soluci\'on limpiadora de cristales para limpiar, no solamente el vidrio de la pantalla sino tambi\'en el gabinete. Hay que ocupar un lienzo libre de pelusa y vaciar el limpiador sobre el lienzo, no sobre el cristal. Lo anterior es muy importante recalcarlo ya que no se debe de introducir el fluido al interior del gabinete, porque podria provocar un corto circuito en el monitor.

\subsubsection*{Teclado}

Es sorprendente la cantidad de suciedad y basura que se puede llegar a acumular en un teclado. La primera l\'inea de defensa es un bote con gas comprimido. Esta operaci\'on de soplado del teclado se debe de realizar en un lugar aparte del sitio donde generalmete trabajo con su computadora, y para evitar que eventualmente este polvo y suciedad regrese. Aunque normalmente no se necesita desarmar el teclado para limpiar el polvo y los desechos que caen sobre el mismo, tal vez se necesite desarmar para limpiar alguna cosa que se haya derramando en \'el.

{\bf Nota.}

Si planea desarmar el teclado y quitar las teclas para limpiar debajo de ellas, es una buena idea hacer una fotocopia de la distribuci\'on del teclado. Puede utilizar posteriormente esta fotocopia para asegurarse de que tenga todas las teclas de vuelta en su posici\'on correcta.


\subsubsection*{Mouse}

Es una buena idea limpiar ocasionalmente el interior de su rat\'on, ya sea normal, o de tipo estacionario. Hay dos clases principales: \'opticos y mec\'anicos.

Los dispositivos mec\'anicos tienen una esfera sin caracter\'isticas especiales que moviliza peque\~nos rodillos a medida que se desplaza el rat\'on en una superficie. El movimiento de los rodillos se traduce en una se\~nal el\'ectrica que pasa a la PC con el tiempo, se va acumulando la suciedad en los rodillos y provoca problemas en el movimiento de la esfera. Se puede utilizar un lienzo de algod\'on o un pa\~no humedecido de alcohol para limpiar los rodillos; o simplemente raspe la materia acumulada con la u\~na de su dedo. Aseg\'urese de quitar la basura del dispositivo antes de que vuelva a colocar la esfera en su lugar.

\end{multicols}
\end{document}
